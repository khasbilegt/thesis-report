\appendix
\addcontentsline{toc}{part}{ХАВСРАЛТ}

\chapter{Сэрээх үг илрүүлэгч}
\label{appendix:wake-word-listener}

\begin{lstlisting}[language=Python]
from precise_runner import PreciseEngine, PreciseRunner
import time
from pixel_ring import pixel_ring # ReSpeaker led controller
import mraa
import os

# Төхөөрөмжийн төлөв илэрхийлэх гэрэл (pixel ring)
en = mraa.Gpio(12)
if os.getuid() != 0:
    time.sleep(1)
en.dir(mraa.DIR_OUT)
en.write(0)
pixel_ring.set_brightness(20)

# Сэрээх\ үг\ илрэхэд\ гүйцэтгэх\ функц
def on_activation():
    print('hello')
    pixel_ring.wakeup() # Төхөөрөмжийн\ гэрлээр\ сэрсэн\ төлөвөө\ илэрхийлнэ
    # ...
    # дуу\ бичих,\ яриа\ таних\ серверт\ илгээх\ гэх\ мэт\ кодууд...
    # ...
    time.sleep(3)
    pixel_ring.off()

def main():
    # Сэрээх\ үг\ илрүүлэгчийг\ эхлүүлэх
    engine = PreciseEngine('precise-engine/precise-engine', 'hey-obi.pb') # hey-obi.pb\ нь\ сургасан\ модель
    runner = PreciseRunner(engine, on_activation=on_activation)
    runner.start()

if __name__ == '__main__':
    main()
    while True:
        time.sleep(1)
\end{lstlisting}

\chapter{Kaldi GStreamer серверт зориулсан тохиргооны файл}
\label{appendix:kaldi-gstreamer-yaml}

\begin{lstlisting}
timeout-decoder : 10
decoder:
    model: /opt/models/tri3b/final.mdl
    lda-mat: /opt/models/tri3b/final.mat
    word-syms: /opt/models/tri3b/graph/words.txt
    fst: /opt/models/tri3b/graph/HCLG.fst
    silence-phones: "1:2:3:4:5"
    left-context: 3
    right-context: 3
out-dir: tmp

use-vad: False
silence-timeout: 5

...
\end{lstlisting}

\chapter{Яриа таних хэсэг}
\label{appendix:stt}

\begin{lstlisting}[language=Python]
def stt():
    audio_file = '.tmp/stt.flac'
    text = False
    # Дууг\ бичиж\ авна
    subprocess.run(['rec', '-c', '1', '-r', '16000', '-d', audio_file,
                    'trim', '0', '15', 'silence', '1', '0.1', '0.3%', '1', '3.0', '0.3%'])
    # Curl\ хэрэгслийг\ ашиглан\ HTTP хүсэлт\ явуулна
    output = str(subprocess.check_output(
        "curl -T {} {}".format(audio_file, conf.config()['STT_URL']), shell=True), 'utf-8')
    print(output)
    if output == 'No workers available':
        return False
    response = json.loads(output)
    print(response)
    # Хэрэв\ төлөв\ нь\ тэгтэй\ тэнцүү\ бол\ амжилттай\ гэж\ үзээд\ бичвэрийг\ буцаана
    if response['status'] == 0:
        text = response['hypotheses'][0]['utterance']
    return text
\end{lstlisting}

\chapter{Чадвар сонгох}
\label{appendix:skill}

\begin{lstlisting}[language=Python]
result = {'obj': None, 'answer': ''}
if intent['intent']['confidence'] > 0.5:
    # intent тохирох\ үйлдлийг\ сонгоно
    # Мэндчилээ
    if intent['intent']['name'] == 'greeting':
        result['answer'] = greeting.greet()
    # Цаг
    elif intent['intent']['name'] == 'get_time':
        result['answer'] = time_n_date.get_time()
    # Огноо
    elif intent['intent']['name'] == 'get_date':
        result['answer'] = time_n_date.get_date()
    # Долоо\ хоногийн\ хэд\ дэх\ өдөр
    elif intent['intent']['name'] == 'get_day':
        result['answer'] = time_n_date.get_day()
    # Цаг\ агаар
    elif intent['intent']['name'] == 'get_weather':
        p = {}
        # entity байвал\ хотын\ нэр\ болон\ огноог\ авч\ функц\ руу\ дамжуулна
        for entity in intent['entities']:
            p[entity['entity']] = entity['value']
        result['answer'] = weather.get_weather(**p)
    # Валютын\ ханш
    elif intent['intent']['name'] == 'get_currency':
        if len(intent['entities']) > 0:
            result['answer'] = currency.get_currency(
                intent['entities'][0]['value'])
        else:
            result['answer'] = currency.get_currency()
    # Захиалга
    elif intent['intent']['name'] == 'order':
        if len(intent['entities']) > 0:
            result['obj'] = order.Order(intent['entities'][0]['value'])
        else:
            result['obj'] = order.Order()
        result['answer'] = 'ямар хэмжээтэй {} захиалах вэ'.format(
            result['obj']._name)
    elif intent['intent']['name'] == 'order_size':
        if len(intent['entities']) > 0:
            result['obj'] = obj.set_size(intent['entities'][0]['value'])
        else:
            result['obj'] = obj.set_size()
        result['answer'] = 'хэдэн ширхэгийг захиалах вэ'
    elif intent['intent']['name'] == 'order_quantity':
        if len(intent['entities']) > 0:
            obj.set_quantity(intent['entities'][0]['value'])
        else:
            obj.set_quantity()
        result['answer'] = obj.order()
else:
    result = False
# Хэрэв\ ямар\ ч\ үйлдэл\ сонгож\ чадаагүй\ бол\ амжилтгүй\ хариу\ буцаана
if result == False:
    result = {'answer': DO_NOT_KNOW[math.floor(
        random.random() * len(DO_NOT_KNOW))], 'obj': obj}
return result
\end{lstlisting}

\chapter{Цаг, огнооны мэдээлэл авах}
\label{appendix:skill-datetime}

\begin{lstlisting}[language=Python]
from datetime import date, datetime

WEEKDAY_NAMES = ['Даваа', 'Мягмар', 'Лхагва', 'Пүрэв', 'Баасан', 'Бямба', 'Ням']

def get_time():
    now = datetime.now()
    hour = now.hour
    minute = now.minute
    return str(hour) + ' цаг ' + str(minute) + ' минут болж байна'

def get_day():
    today = date.today()
    day = WEEKDAY_NAMES[today.weekday()]
    return 'Өнөөдөр ' + day + ' гараг'

def get_date():
    today = date.today()
    month = today.month
    day = today.day
    return 'Өнөөдөр ' + str(month) + ' сарын ' + str(day) + '-ны өдөр'
\end{lstlisting}

\chapter{Яриа үүсгүүр}
\label{appendix:tts}
\begin{lstlisting}[language=Python]
# Яриа\ үүсгүүрээс\ аудио\ файлыг\ авах
def get_wav(sentence=''):
    url = config()['TTS_URL'] + str(sentence)
    r = requests.get(url)
    with open('.tmp/tts.wav', 'wb') as f:
        f.write(r.content)
    return './.tmp/tts.wav'

# Аудио\ файл\ тоглуулах
def play_wav(location='.tmp/tts.wav'):
    chunk = 1024
    f = wave.open(location, "rb")
    p = pyaudio.PyAudio()
    stream = p.open(format=p.get_format_from_width(f.getsampwidth()), channels=f.getnchannels(), rate=f.getframerate(), output=True)
    data = f.readframes(chunk)
    while data:
        stream.write(data)
        data = f.readframes(chunk)
    stream.stop_stream()
    stream.close()
    p.terminate()
    \end{lstlisting}