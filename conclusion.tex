\chapter{Дүгнэлт}

Энэхүү судалгааны ажлаар бусад хэл дээрх гар бичмэл танилттай холбоотой судалгаануудын ажлууд дээр тулгуурлан өөрийн хэлний гар бичмэл танилтанд зориулсан нээлттэй сан, түүнд ашиглах програм хангамжийг хөгжүүлэхээр зорьлоо. Монгол хэл дээрх гар бичмэл танилт нь оюун ухаан, машин сургалтын салбарын энгийн судалгааны ажил биш бөгөөд бичиг баримтыг дижиталжуулах, хүний бичгийн хэвийг ялган таних (authentication), баталгаажуулах (authorization) гэх мэт гар бичмэлтэй холбоотой асар олон төрлийн ажил, судалгаа, програм хангамжуудад ч мөн ашиглагдах боломжтой. Тийм ч учраас дээрх салбаруудтай холбоотой ажил, судалгаа хийхээр зорьж буй хүмүүст энэхүү судалгааны ажил суурь судалгаа нь болох бөгөөд хамгийн чухал нь Монгол хэлний гар бичмэлтэй холбоотой судалгаануудад ашиглагдах нэгдсэн нэг өгөгдлийн сан болон цаашид тогтмол хөгжүүлэгдэж байх хэрэгтэй.

Нийт өгөгдлийн тоо хэмжээ, төрлүүдийг нэмэх, өргөжүүлэхийн тулд python хэл дээрх програм хангамжийн боломжуудыг нэмэх, бичвэр агуулах маягтыг динамик байдлаар үүсгэх, ажиллагааг хурдасгах, нийтээс маягтын зургаа авч автоматаар боловсруулан сангаа шинэчлэх бие даасан вэбсайт, систем гэх мэт ажлууд нэмэгдэж хийх боломжтой.