\begin{abstract}
	Өнөөдөр хиймэл оюун ухаан болон машин сургалтын салбаруудын технологи, судалгаанууд хөгжихийн хэрээр, өмнө нь хүний төсөөлж байгаагүй олон шинэ боломжууд, зөвхөн технологийн салбаруудаар хязгаарлагдалгүй ар араасаа нээгдэж байгаагаас гадна өнөөдөр нийтэд ашиглагдаж буй технологиудын нарийвчлал, үр дүн, чанар ч мөн үүнийг дагаад сайжирч байгаа билээ. Энэ дундаас гар бичмэл танилттай холбоотой судалгаанууд олноор хөгжүүлэгдэх нь бидний мэдэх хүн компьтерийн харилцааг эергээр өөрчилж байна.

	\setcounter{secnumdepth}{0}

	\section{Зорилго}
	Энэ чиглэлийн судалгаанууд, түүн дээр суурилсан програм хангамж, үйлчилгээ, бүтээгдэхүүн хөгжүүлэгдэхэд гар бичмэлийг таних, машин сургалтанд ашиглах өгөгдөл зайлшгүй хэрэгтэй, мөн Монгол хэл дээр энэ төрлийн нээлттэй, нэгдсэн өгөгдлийн сан хомс учир энэхүү судалгааны ажлаар өөрийн эх хэл дээрх нээлттэй гар бичмэлийн санг бий болгох, санг үүсгэхэд ашиглах арга, хэрэгслийг тодорхойлохыг зорьсон юм.

	\section{Зорилт}
	Монгол хэл дээрх гар бичмэлийн нээлттэй санг үүсгэхдээ дараах үе шатын дагуу ажиллана.
	\begin{enumerate}
		\item Үндсэн шаардлагуудыг, аргуудыг тодорхойлох
		\item Тэмдэгтүүд бичүүлж авах маягтын загварыг гаргах
		\item Бөглөгдсөн маягт дээр боловсруулалт хийх, тэмдэгтүүдийг ялгаж авах
		\item Их хэмжээний оролтын үед боловсруулалт хийх боломжтой байдлаар програмыг хөгжүүлэх
		\item Программын нээлттэй авч ашиглах боломжийг хангахын тулд эх код, програмыг GitHub, PyPi дээр байршуулах
	\end{enumerate}

	\setcounter{secnumdepth}{2}

\end{abstract}
