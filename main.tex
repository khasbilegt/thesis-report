%----------------------------------------------------------------------------------------
%   Доорх хэсгийг өөрчлөх шаардлагагүй
%----------------------------------------------------------------------------------------
% !TEX TS-program = xelatex
% !TEX encoding = UTF-8 Unicode
\documentclass[12pt,a4paper]{report}

% \usepackage{fontspec,xltxtra,xunicode}
% \usepackage{fontspec}
% \setmainfont{Times New Roman}
% \setsansfont{Arial}

% \usepackage[T2A]{fontenc}
% \usepackage[utf8]{inputenc}
% \usepackage[english, mongolian]{babel}

\usepackage{polyglossia}
\setmainlanguage{mongolian}
\setotherlanguage{english}
\setmainfont{Times New Roman}
\setsansfont{Arial}
\setromanfont{Times New Roman} 
\newfontfamily{\cyrillicfont}{Times New Roman}[Ligatures=TeX]
\newfontfamily{\cyrillicfontrm}{Times New Roman}
\newfontfamily{\cyrillicfonttt}{Courier New}
\newfontfamily{\cyrillicfontsf}{Arial}
\setkeys{mongolian}{babelshorthands=true}

%\usepackage{natbib}
\usepackage{geometry}
%\usepackage{fancyheadings} fancyheadings is obsolete: replaced by fancyhdr. JL
\usepackage{fancyhdr}
\usepackage{float}
\usepackage{afterpage}
\usepackage{graphicx}
\usepackage{amsmath,amssymb,amsbsy}
\usepackage{dcolumn,array}
\usepackage{tocloft}
\usepackage{dics}
\usepackage{nomencl}
\usepackage{upgreek}
\newcommand{\argmin}{\arg\!\min}
\usepackage{mathtools}
\usepackage[hidelinks, unicode]{hyperref}
\pdfstringdefDisableCommands{\let\uppercase\relax}

\usepackage{algorithm}
\usepackage{algpseudocode}

\usepackage{listings}
\DeclarePairedDelimiter\abs{\lvert}{\rvert}%
\makeatletter
\usepackage{subcaption}
\usepackage[justification={centering}]{caption}
\captionsetup[table]{belowskip=0.5pt}
\usepackage{subfiles}
\usepackage[bottom]{footmisc}
\usepackage{listings}
\renewcommand{\lstlistingname}{Код}
\renewcommand{\lstlistlistingname}{\lstlistingname ын жагсаалт}
\usepackage[final]{pdfpages}

\usepackage{color}
\definecolor{codegreen}{rgb}{0,0.6,0}
\definecolor{codegray}{rgb}{0.5,0.5,0.5}
\definecolor{codepurple}{rgb}{0.58,0,0.82}
\definecolor{backcolour}{rgb}{0.99,0.99,0.99}
 
\lstdefinestyle{mystyle}{
    basicstyle=\ttfamily\small,
    backgroundcolor=\color{backcolour},   
    commentstyle=\color{codegreen},
    keywordstyle=\color{magenta},
    numberstyle=\tiny\color{codegray},
    stringstyle=\color{codepurple},
    %basicstyle=\footnotesize,
    breakatwhitespace=false,
    breaklines=true,
    captionpos=b,
    keepspaces=false,
    numbers=left,
    numbersep=10pt,
    showspaces=false,
    showstringspaces=false,
    showtabs=false,
    tabsize=2
}
 
\lstset{style=mystyle, label=DescriptiveLabel} 

\let\oldabs\abs
\def\abs{\@ifstar{\oldabs}{\oldabs*}}
\makenomenclature
\begin{document}


%----------------------------------------------------------------------------------------
%   Өөрийн мэдээллээ оруулах хэсэг
%----------------------------------------------------------------------------------------

% Дипломийн ажлын сэдэв
\title{ГАР БИЧМЭЛ ТАНИЛТАНД ЗОРИУЛСАН МУИС-Н НЭЭЛТТЭЙ ӨГӨГДӨЛ БЭЛТГЭХ НЬ}
% Дипломын ажлын англи нэр
\titleEng{Preparation of open NUM handwritten character dataset}
% Өөрийн овог нэрийг бүтнээр нь бичнэ
\author{Цэрэнбямбаагийн Хасбилэгт}
% Өөрийн овгийн эхний үсэг нэрээ бичнэ
\authorShort{Ц.Хасбилэгт}
% Удирдагчийн зэрэг цол овгийн эхний үсэг нэр
\supervisor{Др. Б.Сувдаа}
% Хамтарсан удирдагчийн зэрэг цол овгийн эхний үсэг нэр
\cosupervisor{Др. Г.Амарсанаа}

% СиСи дугаар 
\sisiId{15B1SEAS0980}
% Их сургуулийн нэр
\university{МОНГОЛ УЛСЫН ИХ СУРГУУЛЬ}
% Бүрэлдэхүүн сургуулийн нэр
\faculty{ХЭРЭГЛЭЭНИЙ ШИНЖЛЭХ УХААН, ИНЖЕНЕРЧЛЭЛИЙН СУРГУУЛЬ}
% Тэнхимийн нэр
\department{МЭДЭЭЛЭЛ, КОМПЬЮТЕРИЙН УХААНЫ ТЭНХИМ}
% Зэргийн нэр
\degreeName{Бакалаврын судалгааны ажил}
% Суралцаж буй хөтөлбөрийн нэр
\programeName{Мэдээллийн технологи (D061303)}
% Хэвлэгдсэн газар
\cityName{Улаанбаатар}
% Хэвлэгдсэн огноо
\gradyear{2020 он}


%----------------------------------------------------------------------------------------
%   Доорх хэсгийг өөрчлөх шаардлагагүй
%----------------------------------------------------------------------------------------
%----------------------Нүүр хуудастай хамаатай зүйлс----------------------------
\pagenumbering{roman}
\makefrontpage
\maketitle

\doublespace

% Decleration
\begin{huge}
	\textbf{Зохиогчийн баталгаа}
\end{huge} \\ \ \\
\doublespace
Миний бие \@author \ "\@title" \ сэдэвтэй судалгааны ажлыг гүйцэтгэсэн болохыг зарлаж дараах зүйлсийг баталж байна:
\begin{itemize}
	\item Ажил нь бүхэлдээ эсвэл ихэнхдээ Монгол Улсын Их Сургуулийн зэрэг горилохоор дэвшүүлсэн болно.
	\item Энэ ажлын аль нэг хэсгийг эсвэл бүхлээр нь ямар нэг их, дээд сургуулийн зэрэг горилохоор оруулж байгаагүй.
	\item Бусдын хийсэн ажлаас хуулбарлаагүй, ашигласан бол ишлэл, зүүлт хийсэн.
	\item Ажлыг би өөрөө (хамтарч) хийсэн ба миний хийсэн ажил, үзүүлсэн дэмжлэгийг дипломын ажилд тодорхой тусгасан.
	\item Ажилд тусалсан бүх эх сурвалжид талархаж байна.
\end{itemize}
\

Гарын үсэг: \underline{\hspace{5cm}}

Огноо: 	\ \ \underline{\hspace{3cm}}

% Гарчгийг автоматаар оруулна
\setcounter{tocdepth}{1}
\tableofcontents

% Зургийн жагсаалтыг автоматаар оруулна
% \addcontentsline{toc}{part}{ЗУРГИЙН ЖАГСААЛТ}
% \listoffigures

% Хүснэгтийн жагсаалтыг автоматаар оруулна
% \addcontentsline{toc}{part}{ХҮСНЭГТИЙН ЖАГСААЛТ}
% \listoftables

% Кодын жагсаалтыг автоматаар оруулна
% \addcontentsline{toc}{part}{КОДЫН ЖАГСААЛТ}
% \lstlistoflistings

% This puts the word "Page" right justified above everything else.
% \newpage
% \addtocontents{lof}{Зураг~\hfill Хуудас \par}

\renewcommand{\cftlabel}{Зураг}


\doublespace
\pagenumbering{arabic}


\begin{abstract}
	Өнөөдөр хиймэл оюун ухаан болон машин сургалтын салбаруудын технологи, судалгаанууд хөгжихийн хэрээр, өмнө нь хүний төсөөлж байгаагүй олон шинэ боломжууд, зөвхөн технологийн салбаруудаар хязгаарлагдалгүй ар араасаа нээгдэж байгаагаас гадна өнөөдөр нийтэд ашиглагдаж буй технологиудын нарийвчлал, үр дүн, чанар ч мөн үүнийг дагаад сайжирч байгаа билээ. Энэ дундаас гар бичмэл танилттай холбоотой судалгаанууд олноор хөгжүүлэгдэх нь бидний мэдэх хүн компьтерийн харилцааг эергээр өөрчилж байна.

	\setcounter{secnumdepth}{-1}

	\section{Зорилго}
	Энэ чиглэлийн судалгаанууд, түүн дээр суурилсан програм хангамж, үйлчилгээ, бүтээгдэхүүн хөгжүүлэгдэхэд гар бичмэлийг таних, машин сургалтанд ашиглах өгөгдөл зайлшгүй хэрэгтэй, мөн Монгол хэл дээр энэ төрлийн нээлттэй, нэгдсэн өгөгдлийн сан хомс учир энэхүү судалгааны ажлаар өөрийн эх хэл дээрх нээлттэй гар бичмэлийн санг бий болгох, санг үүсгэхэд ашиглах арга, хэрэгслийг тодорхойлохыг зорьсон юм.

	\section{Зорилт}
	Монгол хэл дээрх гар бичмэлийн нээлттэй санг үүсгэхдээ дараах үе шатын дагуу ажиллана.
	\begin{enumerate}
		\item Үндсэн шаардлагуудыг, аргуудыг тодорхойлох
		\item Тэмдэгтүүд бичүүлж авах маягтын загварыг гаргах
		\item Бөглөгдсөн маягт дээр боловсруулалт хийх, тэмдэгтүүдийг ялгаж авах
		\item Их хэмжээний оролтын үед боловсруулалт хийх боломжтой байдлаар програмыг хөгжүүлэх
		\item Программын нээлттэй авч ашиглах боломжийг хангахын тулд эх код, програмыг GitHub, PyPi дээр байршуулах
	\end{enumerate}

	\section{Судалгаа}

	Машин сургалт тэр дундаа гар бичмэл танилттай холбоотой машин сургалтын өгөгдлийг хэрхэн бэлтгэх, зохион байгуулах тал дээр бодитой, албан ёсны судалгааны ажлууд хомс байсан тул өгөгдлийн хэлбэр, хэмжээ, санг хэрхэн зохиомжлох зэргийг энэ чиглэлийн өөр бусад хэл дээрх түгээмэл ашиглагддаг нээлттэй өгөгдлүүдийг судалсаны үндсэн дээр тэдгээрийн жишигт тааруулан гүйцэтгэхээр шийдсэн.

	\section{Аргын тодорхойлолт}

	Энэхүү ажлын мөн чанар нь үндсэн судалгаа бөгөөд гэсэн хэдий ч үр дүн, түүнтэй хамт ашиглагдах програм болон санг үүсгэхдээ өөрийн тодорхойлсон дараах шаардлагуудад нийцүүлэн хөгжүүлэхээр зорьсон. Үүнд:

	\begin{enumerate}
		\item Өгөгдөл нь цэвэр гараар бичигдсэн ба маш сайн боловсруулагдсан байх \label{criteria:1}
		\item Тэмдэгтийн зургууд дахин давтагдаагүй, мөн тухайн тэмдэгтийн бичигдэж болох аль болох олон хувилбаруудыг агуулсан байх \label{criteria:2}
		\item Тэмдэгтүүд дээр шинэ зургууд нэмж өгөгдлийн хэмжээг томруулах, зөвхөн шаардлагатай тэмдэгтүүдээр өөр шинэ датасет үүсгэх боломжтой байх \label{criteria:3}
		\item Өгөгдлийг аль болох олон төрлөөр авч ашиглах, шаардлагатай статистик мэдээллийг авах боломжтой байх \label{criteria:4}
		\item Өгөгдөл нь өөрөө болон, өгөгдөл үүсгэх программ нь хэнд ч авч ашиглахад нээлттэй байх \label{criteria:5}
	\end{enumerate}


	\section{Хөгжүүлэлт}

	Өөрийн тодорхойлсон шаардлага, аргын дагуу хөгжүүлэлтийг хийхдээ дараах технологиудыг авч ашигласан.

	\begin{itemize}
		\item OpenCV --- Анх 2000 онд танилцуулагдсан энэхүү нээлттэй эхийн сан нь өнөөдөр компьютерийн хараа\footnote{ Компьютерийн хараа (= Computer Vision)} -тай холбоотой ажил, судалгаануудад хамгийн өргөн ашиглагддаг бөгөөд зураг боловсруулалтанд энэхүү сангийн санал болгосон функц, аргуудыг ашиглана.
		\item Python --- OpenCV санг Python болон C++ хэл дээр ашиглах боломжтой байдаг бөгөөд зураг боловсруулах програм нь хөгжүүлэлтийн болон нийтэд нээлттэй тавих боломжийг үндэслэн Python хэлийг сонгон ашиглана.
		\item PyPi\footnote{The Python Package Index (PyPI) --- \url{https://pypi.org/}} --- Зураг боловсруулж буй программаа энэхүү платформ дээр байршуулах ба хүссэн хэн нь ч Python багц удирдагч ашиглан амархан татан авах, ашиглах боломжтой болно.
	\end{itemize}

	\section{Програм хангамж}

	Өөрийн ажлын судалгаан дээр үндэслэн өгөгдөл боловсруулах, сан үүсгэхээр хөгжүүлсэн програмууд болон тэдгээрийг хэрхэн ашиглах талаар авч үзнэ. Энэхүү ажил нь програмтай судалгаа буюу судалгаа голлосон, шаардлагтай байдлаар туслах програмуудыг хөгжүүлсэн учир удирдагч багшийн зөвлөснөөр шинжилгээ, зохиомж шаардлагагүйгээр CLI буюу комманд мөр интерфэйстэйгээр хийж гүйцэтгэсэн.


	\section{Үр дүн}

	Энэхүү судалгааны ажлаар бусад хэл дээрх гар бичмэл танилттай холбоотой судалгаануудын ажлууд дээр тулгуурлан өөрийн хэлний гар бичмэл танилтанд зориулсан нээлттэй сан, түүнд ашиглах програм хангамжийг хөгжүүлэхээр зорьлоо. Энэ судалгааны ажил нь цаашид гар бичмэл танилт болон энэ чиглэлийн судалгаа, програм хангамж, үйлчилгээ зэргийг хөгжүүлэхэд суурь болж өгнө гэдэгт итгэлтэй байна.


	Нийт өгөгдлийн тоо хэмжээ, төрлүүдийг нэмэх, өргөжүүлэхийн тулд python хэл дээрх програм хангамжийн\footnote{Эх код: \url{https://github.com/khasbilegt/numiner}}\footnote{PyPi дээрх сан: \url{https://pypi.org/project/numiner/}} боломжуудыг тогтмол нэмэх, сайжруулах, бичвэр агуулах маягтыг динамик байдлаар үүсгэх, ажиллагааг хурдасгах, нийтээс маягтын зургаа авч автоматаар боловсруулан сангаа шинэчлэх бие даасан вэбсайт, систем гэх мэт ажлууд нэмэгдэж хийх шаардлагатай.


\end{abstract}


\addcontentsline{toc}{part}{БҮЛГҮҮД}

\subfile{chapters/research.tex}
\subfile{chapters/methodology.tex}
\subfile{chapters/development.tex}


\chapter{Дүгнэлт}

Энэхүү судалгааны ажлаар бусад хэл дээрх гар бичмэл танилттай холбоотой судалгаануудын ажлууд дээр тулгуурлан өөрийн хэлний гар бичмэл танилтанд зориулсан нээлттэй сан, түүнд ашиглах програм хангамжийг хөгжүүлэхээр зорьлоо. Энэ судалгааны ажил нь цаашид гар бичмэл танилт болон энэ чиглэлийн судалгаа, програм хангамж, үйлчилгээ зэргийг хөгжүүлэхэд суурь болж өгнө гэдэгт итгэлтэй байна.


Нийт өгөгдлийн тоо хэмжээ, төрлүүдийг нэмэх, өргөжүүлэхийн тулд python хэл дээрх програм хангамжийн\footnote{Эх код: \url{https://github.com/khasbilegt/numiner}}\footnote{PyPi дээрх сан: \url{https://pypi.org/project/numiner/}} боломжуудыг нэмэх, бичвэр агуулах маягтыг динамик байдлаар үүсгэх, ажиллагааг хурдасгах, нийтээс маягтын зургаа авч автоматаар боловсруулан сангаа шинэчлэх бие даасан вэбсайт, систем гэх мэт ажлууд нэмэгдэж хийх шаардлагатай.

\renewcommand\bibname{Ашигласан материал}
\addcontentsline{toc}{part}{НОМ ЗҮЙ}
\begin{thebibliography}{99}
	\bibitem{nist19}
	NIST Special Database 19
	\\\url{https://www.nist.gov/srd/nist-special-database-19}

	\bibitem{emnist}
	Cohen, G., Afshar, S., Tapson, J., \& van Schaik, A. (2017).
	\\\textit{EMNIST: an extension of MNIST to handwritten letters.}
	\\\url{http://arxiv.org/abs/1702.05373}

	\bibitem{mnist}
	MNIST --- {W}ikipedia{,} The Free Encyclopedia
	\\\url{https://en.wikipedia.org/wiki/MNIST_database}

	\bibitem{mnist-paper}
	Y. Lecun, L. Bottou, Y. Bengio and P. Haffner
	\textit{Gradient-based learning applied to document recognition} in Proceedings of the IEEE, vol. 86, no. 11, pp. 2278-2324, Nov. 1998, doi: 10.1109/5.726791.

	\bibitem{lob-corpus}
	S. Johansson, G.N. Leech, H. Goodluck: \textit{Manual of information to accompany the Lancaster-Oslo/Bergen corpus of British English, for use with digital computers.} Department of English, University of Oslo, Oslo, (1978)

	\bibitem{iam-database}
	Marti, U., Bunke, H. \textit{The IAM-database: an English sentence database for offline handwriting recognition.} IJDAR 5, 39–46 (2002). https://doi.org/10.1007/s100320200071

	\bibitem{etl}
	ETLCD --- ETL Character Database
	\\\url{http://etlcdb.db.aist.go.jp/obtaining-etl-character-database}

	\bibitem{image-processing}
	Digital image processing --- {W}ikipedia{,} The Free Encyclopedia
	\\\url{https://en.wikipedia.org/wiki/Digital_image_processing}

	\bibitem{opencv}
	Bradski, G. (2000). The OpenCV Library. Dr. Dobb's Journal of Software Tools.
	\\\url{https://opencv.org/}

	\bibitem{python}
	Van Rossum, G., Drake Jr, F. L. (2009). \textit{Python 3 Reference Manual. Scotts Valley,} CA: CreateSpace.
	\\\url{http://www.python.org}


\end{thebibliography}


\appendix
\addcontentsline{toc}{part}{ХАВСРАЛТ}

\chapter{Сэрээх үг илрүүлэгч}
\label{appendix:wake-word-listener}

\begin{lstlisting}[language=Python]
from precise_runner import PreciseEngine, PreciseRunner
import time
from pixel_ring import pixel_ring # ReSpeaker led controller
import mraa
import os

# Төхөөрөмжийн төлөв илэрхийлэх гэрэл (pixel ring)
en = mraa.Gpio(12)
if os.getuid() != 0:
    time.sleep(1)
en.dir(mraa.DIR_OUT)
en.write(0)
pixel_ring.set_brightness(20)

# Сэрээх\ үг\ илрэхэд\ гүйцэтгэх\ функц
def on_activation():
    print('hello')
    pixel_ring.wakeup() # Төхөөрөмжийн\ гэрлээр\ сэрсэн\ төлөвөө\ илэрхийлнэ
    # ...
    # дуу\ бичих,\ яриа\ таних\ серверт\ илгээх\ гэх\ мэт\ кодууд...
    # ...
    time.sleep(3)
    pixel_ring.off()

def main():
    # Сэрээх\ үг\ илрүүлэгчийг\ эхлүүлэх
    engine = PreciseEngine('precise-engine/precise-engine', 'hey-obi.pb') # hey-obi.pb\ нь\ сургасан\ модель
    runner = PreciseRunner(engine, on_activation=on_activation)
    runner.start()

if __name__ == '__main__':
    main()
    while True:
        time.sleep(1)
\end{lstlisting}

\chapter{Kaldi GStreamer серверт зориулсан тохиргооны файл}
\label{appendix:kaldi-gstreamer-yaml}

\begin{lstlisting}
timeout-decoder : 10
decoder:
    model: /opt/models/tri3b/final.mdl
    lda-mat: /opt/models/tri3b/final.mat
    word-syms: /opt/models/tri3b/graph/words.txt
    fst: /opt/models/tri3b/graph/HCLG.fst
    silence-phones: "1:2:3:4:5"
    left-context: 3
    right-context: 3
out-dir: tmp

use-vad: False
silence-timeout: 5

...
\end{lstlisting}

\chapter{Яриа таних хэсэг}
\label{appendix:stt}

\begin{lstlisting}[language=Python]
def stt():
    audio_file = '.tmp/stt.flac'
    text = False
    # Дууг\ бичиж\ авна
    subprocess.run(['rec', '-c', '1', '-r', '16000', '-d', audio_file,
                    'trim', '0', '15', 'silence', '1', '0.1', '0.3%', '1', '3.0', '0.3%'])
    # Curl\ хэрэгслийг\ ашиглан\ HTTP хүсэлт\ явуулна
    output = str(subprocess.check_output(
        "curl -T {} {}".format(audio_file, conf.config()['STT_URL']), shell=True), 'utf-8')
    print(output)
    if output == 'No workers available':
        return False
    response = json.loads(output)
    print(response)
    # Хэрэв\ төлөв\ нь\ тэгтэй\ тэнцүү\ бол\ амжилттай\ гэж\ үзээд\ бичвэрийг\ буцаана
    if response['status'] == 0:
        text = response['hypotheses'][0]['utterance']
    return text
\end{lstlisting}

\chapter{Чадвар сонгох}
\label{appendix:skill}

\begin{lstlisting}[language=Python]
result = {'obj': None, 'answer': ''}
if intent['intent']['confidence'] > 0.5:
    # intent тохирох\ үйлдлийг\ сонгоно
    # Мэндчилээ
    if intent['intent']['name'] == 'greeting':
        result['answer'] = greeting.greet()
    # Цаг
    elif intent['intent']['name'] == 'get_time':
        result['answer'] = time_n_date.get_time()
    # Огноо
    elif intent['intent']['name'] == 'get_date':
        result['answer'] = time_n_date.get_date()
    # Долоо\ хоногийн\ хэд\ дэх\ өдөр
    elif intent['intent']['name'] == 'get_day':
        result['answer'] = time_n_date.get_day()
    # Цаг\ агаар
    elif intent['intent']['name'] == 'get_weather':
        p = {}
        # entity байвал\ хотын\ нэр\ болон\ огноог\ авч\ функц\ руу\ дамжуулна
        for entity in intent['entities']:
            p[entity['entity']] = entity['value']
        result['answer'] = weather.get_weather(**p)
    # Валютын\ ханш
    elif intent['intent']['name'] == 'get_currency':
        if len(intent['entities']) > 0:
            result['answer'] = currency.get_currency(
                intent['entities'][0]['value'])
        else:
            result['answer'] = currency.get_currency()
    # Захиалга
    elif intent['intent']['name'] == 'order':
        if len(intent['entities']) > 0:
            result['obj'] = order.Order(intent['entities'][0]['value'])
        else:
            result['obj'] = order.Order()
        result['answer'] = 'ямар хэмжээтэй {} захиалах вэ'.format(
            result['obj']._name)
    elif intent['intent']['name'] == 'order_size':
        if len(intent['entities']) > 0:
            result['obj'] = obj.set_size(intent['entities'][0]['value'])
        else:
            result['obj'] = obj.set_size()
        result['answer'] = 'хэдэн ширхэгийг захиалах вэ'
    elif intent['intent']['name'] == 'order_quantity':
        if len(intent['entities']) > 0:
            obj.set_quantity(intent['entities'][0]['value'])
        else:
            obj.set_quantity()
        result['answer'] = obj.order()
else:
    result = False
# Хэрэв\ ямар\ ч\ үйлдэл\ сонгож\ чадаагүй\ бол\ амжилтгүй\ хариу\ буцаана
if result == False:
    result = {'answer': DO_NOT_KNOW[math.floor(
        random.random() * len(DO_NOT_KNOW))], 'obj': obj}
return result
\end{lstlisting}

\chapter{Цаг, огнооны мэдээлэл авах}
\label{appendix:skill-datetime}

\begin{lstlisting}[language=Python]
from datetime import date, datetime

WEEKDAY_NAMES = ['Даваа', 'Мягмар', 'Лхагва', 'Пүрэв', 'Баасан', 'Бямба', 'Ням']

def get_time():
    now = datetime.now()
    hour = now.hour
    minute = now.minute
    return str(hour) + ' цаг ' + str(minute) + ' минут болж байна'

def get_day():
    today = date.today()
    day = WEEKDAY_NAMES[today.weekday()]
    return 'Өнөөдөр ' + day + ' гараг'

def get_date():
    today = date.today()
    month = today.month
    day = today.day
    return 'Өнөөдөр ' + str(month) + ' сарын ' + str(day) + '-ны өдөр'
\end{lstlisting}

\chapter{Яриа үүсгүүр}
\label{appendix:tts}
\begin{lstlisting}[language=Python]
# Яриа\ үүсгүүрээс\ аудио\ файлыг\ авах
def get_wav(sentence=''):
    url = config()['TTS_URL'] + str(sentence)
    r = requests.get(url)
    with open('.tmp/tts.wav', 'wb') as f:
        f.write(r.content)
    return './.tmp/tts.wav'

# Аудио\ файл\ тоглуулах
def play_wav(location='.tmp/tts.wav'):
    chunk = 1024
    f = wave.open(location, "rb")
    p = pyaudio.PyAudio()
    stream = p.open(format=p.get_format_from_width(f.getsampwidth()), channels=f.getnchannels(), rate=f.getframerate(), output=True)
    data = f.readframes(chunk)
    while data:
        stream.write(data)
        data = f.readframes(chunk)
    stream.stop_stream()
    stream.close()
    p.terminate()
    \end{lstlisting}

\end{document}
